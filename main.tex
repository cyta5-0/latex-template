\documentclass[a4paper,12pt]{article}
\usepackage[utf8]{inputenc}
\usepackage{graphicx}
\usepackage{hyperref}
\usepackage{geometry}
\usepackage{authblk}
% metadata.tex - Metadatos para artículos en LaTeX con RDFa/Schema.org

% Título y autores
\title{Título del artículo en LaTeX con Metadatos Semánticos}  
\author{Nombre del Autor\thanks{ORCID: 0000-0000-0000-0000, Afiliación}}  
\date{\today}  

% Palabras clave
\keywords{Palabra clave 1, Palabra clave 2, Palabra clave 3}  

% Licencia Creative Commons
\usepackage{ccicons}
\newcommand{\licencia}{\ccby \ccnc \ccsa}

% Metadatos RDFa / Schema.org
\usepackage{xmpincl} % Para incluir metadatos en PDF
\includexmp{metadata} % Archivo RDF/XML externo con datos estructurados


\geometry{left=2.5cm, right=2.5cm, top=3cm, bottom=3cm}

% Paquete para referencias en formato APA
\usepackage[style=apa,sorting=nyt]{biblatex}
\addbibresource{references.bib} % Archivo .bib con las referencias

%\title{Título del Informe}
%\author[1]{Nombre del Autor 1}
%\author[2]{Nombre del Autor 2}
%\affil[1]{Institución 1, Correo Autor 1}
%\affil[2]{Institución 2, Correo Autor 2}
%\date{\today}
\title{\documentTitle}
\author{\authorName \\ \affiliation}
\date{\today}


\begin{document}

\maketitle

\begin{abstract}
Este es el resumen del informe, donde se presenta una visión general del contenido y los principales hallazgos.

\textbf{Keywords:} keyword 1; keyword 2; keyword 3, keyword 4; keyword 5
\end{abstract}

\section{Introducción}
Aquí se describe el contexto del informe, los objetivos, como resultados a priori, y la motivación detrás del estudio.
Además se especifica la audiencia para la cual se destina el informe (quién se espera que lo lea y actúe según recomendaciones y conclusiones).
La función principal de la introducción es que:
\begin{itemize}
\item introduce el tema,
\item determina su importancia,
\item establece una revisión de sus antecedentes históricos en la problemática tratada a través de la literatura científica y tecnológica pertinente,
\item explicita las teorías subyacentes del tema,
\item indica el propósito o el motivo de la investigación,
\item establece el alcance y los límites de la investigación, y
\item establece la forma en que se planea desarrollar la problemática para alcanzar los objetivos.


\end{itemize}

\section{Metodología}
Explicación de los métodos utilizados para llevar a cabo la investigación. Consecuentemente se realiza una descripción, lo suficientemente detallada y completa, de los procedimientos utilizados en la investigación, que les permita a los lectores evaluar la forma en que los resultados fueron obtenidos.
Si la investigación incluía aparatos, instrumentos o reactivos, se requerirá de una descripción de ellos, su diseño y la precisión de los instrumentos.

\section{Resultados}
Presentación de los hallazgos obtenidos en la investigación, basados en los métodos establecidos y que serán esenciales para fundamentar las conclusiones. 


\section{Discusión}
El autor interpreta los resultados más relevantes obtenidos que ya se han fundamentado en dicha sección; y por lo tanto no introduce ningún nuevo material. Se constituye en una síntesis de los puntos clave del informe, y se establecen posibles direcciones futuras.

\section{Conclusión}
El autor interpreta los resultados más relevantes obtenidos.

\section{Notas}
Esta plantilla fue realizada en base al recuso publicado por CyTA titulado: Informes científicos y técnicos: detalles para su elaboración \cite{Perissé2019}; y a la revisión de literatura y desarrollos técnicos de ChatGPT\cite{ChatGPT2025}

.


\printbibliography % Imprime las referencias automáticamente desde el archivo .bib


\end{document}
