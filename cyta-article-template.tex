% Plantilla para Artículos Científicos en LaTeX - CyTA
% Basado en la plantilla original con mejoras para integración con GitHub y curación semántica

\documentclass[a4paper,12pt]{article}
\usepackage[utf8]{inputenc}
\usepackage[T1]{fontenc}
\usepackage{graphicx}
\usepackage{float}
\usepackage{hyperref}
\usepackage{apacite} % Citas en APA
\usepackage{lmodern}
\usepackage{geometry}
\geometry{margin=1in}
\usepackage{amsmath, amssymb, amsfonts}
\usepackage{listings} % Para código fuente

% Metadatos RDFa y Schema.org en LaTeX
\hypersetup{
    pdfauthor={Nombre del Autor},
    pdftitle={Título del Artículo},
    pdfsubject={Disciplina científica},
    pdfkeywords={Palabras clave, CyTA, Web Semántica},
    pdfproducer={LaTeX con BibTeX},
    pdfcreator={pdfLaTeX}
}

\title{\textbf{Título del Artículo Científico}}
\author{
    Nombre del Autor $^{1,}$\footnote{ORCID: 0000-0000-0000-0000} \\
    \small $^1$Afiliación Institucional, Ciudad, País \\
    \small \texttt{email@ejemplo.com}
}
\date{\today}

\begin{document}

\maketitle

\begin{abstract}
    Aquí va el resumen del artículo. Debe proporcionar una visión general del contenido y los hallazgos principales.
    \textbf{Palabras clave}: palabra1, palabra2, palabra3.
\end{abstract}

\section{Introducción}
Texto introductorio...

\section{Metodología}
Texto de metodología...

\section{Resultados}
Texto de resultados...

\section{Discusión}
Texto de discusión...

\section{Conclusión}
Texto de conclusión...

\bibliographystyle{apacite}
\bibliography{referencias} % Archivo BibTeX con referencias

\end{document}
